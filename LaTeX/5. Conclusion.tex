\section{Conclusion and future work}
The objective of this research was to analyse the effect of local and global rarefaction zones on the heat transfer onto simple shapes. 

In \autoref{1} a significant increase in the interest in space exploration was highlighted, and the need for new aerodynamic decelerators was evidenced. Moreover, it was pointed out that research into hypersonic rarefied flows is needed to develop the aforementioned aeroshells. 

In \autoref{section:2} the theoretical background for rarefied flow was outlined, and relevant research in the field examined. It was discovered that the optimal way to model this flow regime is through Direct Simulation Monte Carlo.

In \autoref{section:3} the working principles of DSMC were outlined, and four set of simulations on simple shapes (varying global Knudsen number, varying edge radius, constant local Knudsen number and varying angle of attack) were designed based on them. Convergence studies were thus conducted and the simulation results were validated, in order to ensure the accuracy of the simulations.

In \autoref{section:4} the main findings were presented: 
\begin{itemize}
    \item A drastic change in flow physics with rarefaction was noted, and a theoretical explanation was provided.
    \item The dependence of stagnation point Stanton number on global Knudsen number was evidenced and verified through comparison with literature.
    \item The dependence of the ratio between peak and stagnation point Stanton number with Knudsen number was investigated, in order to determine a correction factor for aerothermodynamic rapid prototyping codes. It was however discovered that further research is required, directed especially towards the continuum flow regime.
\end{itemize}

Overall, the research conducted can be considered successful regarding the investigation into the effects of rarefaction. More work is however needed to obtain a valid correction factor.